\documentclass[a4paper]{book}
\usepackage[times,inconsolata,hyper]{Rd}
\usepackage{makeidx}
\usepackage[utf8]{inputenc} % @SET ENCODING@
% \usepackage{graphicx} % @USE GRAPHICX@
\makeindex{}
\begin{document}
\chapter*{}
\begin{center}
{\textbf{\huge Package `QALY'}}
\par\bigskip{\large \today}
\end{center}
\begin{description}
\raggedright{}
\inputencoding{utf8}
\item[Type]\AsIs{Package}
\item[Title]\AsIs{Calculate QALY Gains with Discounting and Inflated Costs}
\item[Version]\AsIs{0.1.0}
\item[Author]\AsIs{Nathan Green}
\item[Maintainer]\AsIs{Nathan Green }\email{nathan.green@imperial.ac.uk}\AsIs{}
\item[Description]\AsIs{S3 package for QALY calculation and plotting.}
\item[License]\AsIs{MIT}
\item[Encoding]\AsIs{UTF-8}
\item[LazyData]\AsIs{true}
\item[Imports]\AsIs{readr,
datapkg}
\item[Suggests]\AsIs{testthat}
\item[RoxygenNote]\AsIs{6.0.1}
\end{description}
\Rdcontents{\R{} topics documented:}
\inputencoding{utf8}
\HeaderA{adjusted\_life\_years}{Adjusted Life Years Object Constructor}{adjusted.Rul.life.Rul.years}
%
\begin{Description}\relax
For use as input to the QALY and DALY functions.
\end{Description}
%
\begin{Usage}
\begin{verbatim}
adjusted_life_years(start_year = 0, end_year = NA, age = NA,
  time_horizon = NA, utility, discount_rate = 0.035)
\end{verbatim}
\end{Usage}
%
\begin{Arguments}
\begin{ldescription}
\item[\code{start\_year}] Calendar year to begin calculation

\item[\code{end\_year}] Calendar year to end calculation

\item[\code{age}] Age at start of period

\item[\code{time\_horizon}] Number of time periods from start to end date

\item[\code{utility}] Proportion health detriment

\item[\code{discount\_rate}] Fixed proportion reduction over time
\end{ldescription}
\end{Arguments}
%
\begin{Value}
Object of class adjusted\_life\_years
\end{Value}
%
\begin{SeeAlso}\relax
\code{\LinkA{QALY}{QALY}}, \code{\LinkA{DALY}{DALY}}
\end{SeeAlso}
%
\begin{Examples}
\begin{ExampleCode}

AdjLifeYears <- adjusted_life_years(
                    start_year = 2016,
                    end_year = 2020,
                    age = 33,
                    time_horizon = NA,
                    utility = 0.9,
                    discount_rate = 0.035)

total_QALYs(AdjLifeYears)
## 2.913622

total_QALY(1)
## "Error: Not an adjusted_life_years class input object."

\end{ExampleCode}
\end{Examples}
\inputencoding{utf8}
\HeaderA{calc\_QALY}{Calculate Quality-Adjusted Life Years}{calc.Rul.QALY}
%
\begin{Description}\relax
Discounted total QALYs upto a defined time horizon.
This is a simpler function.
An alternative Method is also available (see \code{\LinkA{total\_QALYs}{total.Rul.QALYs}}).
\end{Description}
%
\begin{Usage}
\begin{verbatim}
calc_QALY(utility = NA, age = NA, time_horizon = NA, halfend = FALSE,
  start_delay = 0)
\end{verbatim}
\end{Usage}
%
\begin{Arguments}
\begin{ldescription}
\item[\code{utility}] Vector of values between 0 and 1 (1 - utility loss)

\item[\code{age}] Year of age

\item[\code{time\_horizon}] Non-negative value how many time step into future as sum limit

\item[\code{halfend}] Should the last year be a half year

\item[\code{start\_delay}] What time delay to origin, to shift discounting
\end{ldescription}
\end{Arguments}
%
\begin{Details}\relax
Uses the following formula, for year  \code{i}:

\deqn{ \sum prop_year(i) * utility(i) * QoL(age(i)) * discount_factor(i) }{}

for  \code{i} = 1, ..., \code{time\_horizon}.

\code{prop\_year} is useful for fractions of years at the start and end of the period.
However, since we may not know this then may not be necessary.
\end{Details}
%
\begin{References}\relax
Sassi, Franco, Health Policy and Planning, 5, 402-408, Calculating QALYs,
comparing QALY and DALY calculations, volume 21, 2006
\end{References}
%
\begin{SeeAlso}\relax
\code{\LinkA{calc\_QALY\_CFR}{calc.Rul.QALY.Rul.CFR}},
\code{\LinkA{calc\_QALY\_population}{calc.Rul.QALY.Rul.population}}
\code{\LinkA{total\_QALYs}{total.Rul.QALYs}}
\end{SeeAlso}
%
\begin{Examples}
\begin{ExampleCode}
calc_QALY(utility = 0.9, age = 13, time_horizon = 49)

\end{ExampleCode}
\end{Examples}
\inputencoding{utf8}
\HeaderA{calc\_QALY\_CFR}{Calculate QALYs using Case-Fatality Rates}{calc.Rul.QALY.Rul.CFR}
%
\begin{Description}\relax
Sum to time of death or a prespecified time horizon.
CFRs are dependent on age.
Utility is fixed over time.
\end{Description}
%
\begin{Usage}
\begin{verbatim}
calc_QALY_CFR(AGES = NA, cfr_age_lookup = NULL, time_horizon = NA,
  utility = 0.9)
\end{verbatim}
\end{Usage}
%
\begin{Arguments}
\begin{ldescription}
\item[\code{AGES}] Vector of ages at start

\item[\code{cfr\_age\_lookup}] Data frame of case-fatality ratios for ages

\item[\code{time\_horizon}] Vector of end dates for each individual

\item[\code{utility}] Proportion health detriment
\end{ldescription}
\end{Arguments}
%
\begin{SeeAlso}\relax
\code{\LinkA{calc\_QALY}{calc.Rul.QALY}}
\end{SeeAlso}
%
\begin{Examples}
\begin{ExampleCode}


# 12 month case fatality rate
# Crofts et al (2008)
cfr_age_breaks <- c(15, 45, 65, 200)
cfr_age_levels <- levels(cut(0, cfr_age_breaks, right = FALSE))

cfr_age_lookup <- data.frame(age = cfr_age_levels,
                             cfr = c(0.0018, 0.0476, 0.1755),
                             a = c(1, 125, 413), #beta distn
                             b = c(564, 2500, 1940))

rownames(cfr_age_lookup) <- cfr_age_lookup$age

# status-quo
QALY.statusquo <- calc_QALY_CFR(AGES = IMPUTED_sample$cfr_age_groups[IMPUTED_sample$uk_tb==1],
                                cfr_age_lookup)

# screened imputed sample
QALY.screened <- calc_QALY_CFR(AGES = IMPUTED_sample$cfr_age_groups[IMPUTED_sample$uk_tb1==1],
                               cfr_age_lookup)

# who changed LTBI status
uk_TB.completedTx <- (IMPUTED_sample$uk_tb1==0 & IMPUTED_sample$uk_tb==1)

# fixed over time
QALY_uk_TB.completedTx <- years(death.isdtt[uk_TB.completedTx] - uk_tb.isdtt[uk_TB.completedTx])

QALY.screened <- c(QALY.screened, QALY_uk_TB.completedTx)

# (non-fixed) discounted all-cause (non-active TB)
notification_to_allcause_death <- (IMPUTED_sample$date_death1_issdt - IMPUTED_sample$fup_issdt)/365

calc_QALY_CFR(time_horizon = notification_to_allcause_death[uk_TB.completedTx],
              utility = 1.0)

\end{ExampleCode}
\end{Examples}
\inputencoding{utf8}
\HeaderA{calc\_QALY\_population}{Calculate QALYs For Population}{calc.Rul.QALY.Rul.population}
%
\begin{Description}\relax
This is a wrapper function for \code{calc\_QALY} over
multiple time horizons (e.g. individuals).
\end{Description}
%
\begin{Usage}
\begin{verbatim}
calc_QALY_population(utility, age, time_horizons, start_delay = 0, ...)
\end{verbatim}
\end{Usage}
%
\begin{Arguments}
\begin{ldescription}
\item[\code{utility}] Vector of utilities for each year in to the future, between 0 and 1

\item[\code{age}] Vector of ages at start

\item[\code{time\_horizons}] Vector of non-negative durations

\item[\code{start\_delay}] What time delay to origin, to shift discounting

\item[\code{...}] Additional arguments
\end{ldescription}
\end{Arguments}
%
\begin{Details}\relax
Assume that the utilities are the same for all individuals.
\end{Details}
%
\begin{Value}
QALY vector
\end{Value}
%
\begin{SeeAlso}\relax
\code{\LinkA{calc\_QALY\_CFR}{calc.Rul.QALY.Rul.CFR}},
\code{\LinkA{calc\_QALY}{calc.Rul.QALY}}
\end{SeeAlso}
\inputencoding{utf8}
\HeaderA{CFR\_time\_horizon.adjusted\_life\_years}{Case-Fatality Rate Determined Time Horizon}{CFR.Rul.time.Rul.horizon.adjusted.Rul.life.Rul.years}
%
\begin{Description}\relax
Using this function then don't require a separate CFR
function to calculate the total QALYs.
DRY priniciple.
\end{Description}
%
\begin{Usage}
\begin{verbatim}
CFR_time_horizon.adjusted_life_years(adjusted_life_years, cfr_modelframe)
\end{verbatim}
\end{Usage}
%
\begin{Arguments}
\begin{ldescription}
\item[\code{adjusted\_life\_years}] An object of class adjusted\_life\_years

\item[\code{cfr\_modelframe}] Data frame with CFR and ages
\end{ldescription}
\end{Arguments}
%
\begin{Value}
An object of class adjusted\_life\_years
\end{Value}
%
\begin{Examples}
\begin{ExampleCode}

AdjLifeYears <- adjusted_life_years(
                    start_year = 2016,
                    end_year = 2020,
                    age = NA,
                    time_horizon = NA,
                    utility = 0.9,
                    discount_rate = 0.035)

cfr_modelframe <- model.frame(cfr ~ age, data = cfr_age_lookup)

CFR_time_horizon.adjusted_life_years(AdjLifeYears, cfr_modelframe)

\end{ExampleCode}
\end{Examples}
\inputencoding{utf8}
\HeaderA{discount}{Discounted Values Over Time}{discount}
%
\begin{Description}\relax
Dscounted value, e.g. cost or health (QALY), for each time point, e.g. year.
E.g. a QALY in the future is worth less to us now because of 'interest'
or conversely we'd need more QALYs now to have a QALY further in the future.
\end{Description}
%
\begin{Usage}
\begin{verbatim}
discount(discount_rate = 0.035, t_limit = 100)
\end{verbatim}
\end{Usage}
%
\begin{Arguments}
\begin{ldescription}
\item[\code{discount\_rate}] Discount factor, default at 3.5\%

\item[\code{t\_limit}] Time period (positive integer) to discount over starting from 1
\end{ldescription}
\end{Arguments}
%
\begin{Details}\relax
Formula used is
\deqn{ 1/(1 + discount_rate)^years}{}
\end{Details}
%
\begin{Value}
Discounted value for each time point up to \code{t\_limit}
\end{Value}
%
\begin{References}\relax
Severens, Johan L and Milne, Richard J,
Value in Health, 4, Discounting Health Outcomes in Economic Evaluation : The Ongoing Debate,
volume 7, 2004
\end{References}
%
\begin{Examples}
\begin{ExampleCode}

D <- discount(t_limit = 10)
utility <- 0.97

# Discounted QALYs upto 10 years
sum(utility * D)

\end{ExampleCode}
\end{Examples}
\inputencoding{utf8}
\HeaderA{fillin\_missing\_utilities}{Fill-in Missing Trailing Utilities}{fillin.Rul.missing.Rul.utilities}
%
\begin{Description}\relax
Repeat last value up to final period.
\end{Description}
%
\begin{Usage}
\begin{verbatim}
fillin_missing_utilities(utility, time_horizon)
\end{verbatim}
\end{Usage}
%
\begin{Arguments}
\begin{ldescription}
\item[\code{utility}] Vector of health quality of life values between 0 and 1

\item[\code{time\_horizon}] Non-negative integer
\end{ldescription}
\end{Arguments}
%
\begin{Value}
Vector of utilities
\end{Value}
%
\begin{SeeAlso}\relax
\code{\LinkA{calc\_QALY\_population}{calc.Rul.QALY.Rul.population}}
\end{SeeAlso}
\inputencoding{utf8}
\HeaderA{inflation\_adjust\_cost}{Calculate Inflation Adjusted Costs}{inflation.Rul.adjust.Rul.cost}
%
\begin{Description}\relax
Up to the present time inflated upwards.
Option to use the ONS GDP\_Deflators\_Qtrly\_National\_Accounts or
a fixed 3.5\%.
Can't download directly into function because the .csv on the website is too messy as-is.
This would be good to do though so that can always use latest version.
TODO: webscraping? regular expressions?
\end{Description}
%
\begin{Usage}
\begin{verbatim}
inflation_adjust_cost(from_year, to_year, from_cost, reference = NA,
  fixed = TRUE)
\end{verbatim}
\end{Usage}
%
\begin{Arguments}
\begin{ldescription}
\item[\code{from\_year}] Date of cost to convert from

\item[\code{to\_year}] Date to convert cost to

\item[\code{from\_cost}] Cost at \code{from\_year}

\item[\code{reference}] Source of data (string)

\item[\code{fixed}] Fixed 3.5\% rate of inflation?
\end{ldescription}
\end{Arguments}
%
\begin{Examples}
\begin{ExampleCode}
from_year <- 2012
to_year <- 2015
from_cost <- 96.140

inflation_adjust_cost(from_year, to_year, from_cost, fixed = FALSE)
#100

inflation_adjust_cost(from_year = 2010, to_year = 2016, from_cost = 1)
#1.229255
1*(1+0.035)^6

\end{ExampleCode}
\end{Examples}
\inputencoding{utf8}
\HeaderA{make\_discount}{Make an Encapsulated Discount Function}{make.Rul.discount}
%
\begin{Description}\relax
This format doesn't need to keep track of time.
\end{Description}
%
\begin{Usage}
\begin{verbatim}
make_discount(discount_rate = 0.035)
\end{verbatim}
\end{Usage}
%
\begin{Arguments}
\begin{ldescription}
\item[\code{discount\_rate}] Discount factor, default at 3.5\%
\end{ldescription}
\end{Arguments}
%
\begin{Value}
Function with global scoped index
\end{Value}
\inputencoding{utf8}
\HeaderA{plot\_QALY}{QALY Plot as a Non-Increasing Step Function Over Time}{plot.Rul.QALY}
%
\begin{Description}\relax
QALY Plot as a Non-Increasing Step Function Over Time
\end{Description}
%
\begin{Usage}
\begin{verbatim}
plot_QALY(QALYs, overlay = FALSE, XLIM = c(0, 80), COL = "light grey",
  age_annotate = FALSE)
\end{verbatim}
\end{Usage}
%
\begin{Arguments}
\begin{ldescription}
\item[\code{QALYs}] QALY class object

\item[\code{overlay}] Overlay lines on current plot?

\item[\code{age\_annotate}] 
\end{ldescription}
\end{Arguments}
%
\begin{SeeAlso}\relax
\code{\LinkA{total\_QALYs.adjusted\_life\_years}{total.Rul.QALYs.adjusted.Rul.life.Rul.years}}
\end{SeeAlso}
%
\begin{Examples}
\begin{ExampleCode}
QALYs <- total_QALYs(AdjLifeYears)
plot_QALY(QALYs)

## list of objects
plot_QALY(QALY_diseasefree[[1]], overlay = F)
map(QALY_diseasefree, plot_QALY, overlay = T, COL = rgb(0, 0, 0, 0.1))
map(QALYloss_diseasefree, plot_QALY, overlay = T, COL = "red")

# cumulative total QALY loss
sapply(QALY_diseasefree,
       FUN = function(x) attr(x, "yearly_QALYs")) %>%
       rowSums(na.rm = TRUE) %>%
       cumsum %>%
       plot(type = 'l')

\end{ExampleCode}
\end{Examples}
\inputencoding{utf8}
\HeaderA{total\_QALYs}{Calculate Life-Time QALYs}{total.Rul.QALYs}
%
\begin{Description}\relax
Calculate Life-Time QALYs
\end{Description}
%
\begin{Usage}
\begin{verbatim}
total_QALYs(adjusted_life_years)
\end{verbatim}
\end{Usage}
%
\begin{Arguments}
\begin{ldescription}
\item[\code{adjusted\_life\_years}] Object of class \code{adjusted\_life\_years}
\end{ldescription}
\end{Arguments}
%
\begin{Value}
QALYs object
\end{Value}
\inputencoding{utf8}
\HeaderA{total\_QALYs.adjusted\_life\_years}{Calculate Life-Time QALYs}{total.Rul.QALYs.adjusted.Rul.life.Rul.years}
%
\begin{Description}\relax
Calculate Life-Time QALYs
\end{Description}
%
\begin{Usage}
\begin{verbatim}
## S3 method for class 'adjusted_life_years'
total_QALYs(adj_years)
\end{verbatim}
\end{Usage}
%
\begin{Arguments}
\begin{ldescription}
\item[\code{adjusted\_life\_years}] Object of class \code{adjusted\_life\_years}
\end{ldescription}
\end{Arguments}
%
\begin{Value}
QALYs object
\end{Value}
%
\begin{Examples}
\begin{ExampleCode}

AdjLifeYears <- adjusted_life_years(
                    start_year = 2016,
                    end_year = 2020,
                    age = NA,
                    time_horizon = NA,
                    utility = 0.9,
                    discount_rate = 0.035)

total_QALYs(AdjLifeYears)
## 2.913622

total_QALYs(1)
## "Error: Not an adjusted_life_years class input object."

\end{ExampleCode}
\end{Examples}
\inputencoding{utf8}
\HeaderA{total\_QALYs.default}{Calculate Life-Time QALYs}{total.Rul.QALYs.default}
%
\begin{Description}\relax
Calculate Life-Time QALYs
\end{Description}
%
\begin{Usage}
\begin{verbatim}
## Default S3 method:
total_QALYs(adjusted_life_years)
\end{verbatim}
\end{Usage}
%
\begin{Arguments}
\begin{ldescription}
\item[\code{adjusted\_life\_years}] Object of class \code{adjusted\_life\_years}
\end{ldescription}
\end{Arguments}
%
\begin{Value}
QALYs object
\end{Value}
\printindex{}
\end{document}
